\documentclass{InsightArticle}

\usepackage[dvips]{graphicx}
\usepackage{color}
\usepackage{listings}
\usepackage{verbatim}
\usepackage{textcomp}

\definecolor{listcomment}{rgb}{0.0,0.5,0.0}
\definecolor{listkeyword}{rgb}{0.0,0.0,0.5}
\definecolor{listnumbers}{gray}{0.65}
\definecolor{listlightgray}{gray}{0.955}
\definecolor{listwhite}{gray}{1.0}


%%%%%%%%%%%%%%%%%%%%%%%%%%%%%%%%%%%%%%%%%%%%%%%%%%%%%%%%%%%%%%%%%%
%
%  hyperref should be the last package to be loaded.
%
%%%%%%%%%%%%%%%%%%%%%%%%%%%%%%%%%%%%%%%%%%%%%%%%%%%%%%%%%%%%%%%%%%
\usepackage[dvips,
bookmarks,
bookmarksopen,
backref,
colorlinks,linkcolor={blue},citecolor={blue},urlcolor={blue},
]{hyperref}


\title{Acquisition of Synchrotron Radiation micro-CT images for the investigation of bone
micro-cracks}


%
% NOTE: This is the last number of the "handle" URL that
% The Insight Journal assigns to your paper as part of the
% submission process. Please replace the number "1338" with
% the actual handle number that you get assigned.
%
\newcommand{\IJhandlerIDnumber}{3261}

\lstset{frame = tb,
       framerule = 0.25pt,
       float,
       fontadjust,
       backgroundcolor={\color{listlightgray}},
       basicstyle = {\ttfamily\footnotesize},
       keywordstyle = {\ttfamily\color{listkeyword}\textbf},
       identifierstyle = {\ttfamily},
       commentstyle = {\ttfamily\color{listcomment}\textit},
       stringstyle = {\ttfamily},
       showstringspaces = false,
       showtabs = false,
       numbers = left,
       numbersep = 6pt,
       numberstyle={\ttfamily\color{listnumbers}},
       tabsize = 2,
       language=[ANSI]C++,
       floatplacement=!h
       }

\release{1.10}

\author{Maria A. Zuluaga$^{1,2}$, Aymeric Larrue$^{3}$, Aline Rattner$^{4}$, \\Laurence Vico$^{4}$, Fran\c{c}oise Peyrin$^{1,2}$}
\authoraddress{$^{1}$CREATIS; Universit\'{e} de Lyon; Universit\'{e} Lyon 1; INSA-Lyon; CNRS UMR5220; INSERM U630; F-69621 Villeurbanne, France\\
$^{2}$ European Synchrotron Radiation Facility, Grenoble, France\\
$^{3}$Institute of Biomedical Engineering, Department of Engineering Science, University of Oxford, \\Oxford, UK\\
$^{4}$LBTO, Université de Lyon, St-Etienne, F-42023 France ; Inserm, U890, \\Saint-Etienne F-42023
St. Etienne, France}

\begin{document}


%
% Add hyperlink to the web location and license of the paper.
% The argument of this command is the handler identifier given
% by the Insight Journal to this paper.
%
\IJhandlefooter{\IJhandlerIDnumber}


\ifpdf
\else
   %
   % Commands for including Graphics when using latex
   %
   \DeclareGraphicsExtensions{.eps,.jpg,.gif,.tiff,.bmp,.png}
   \DeclareGraphicsRule{.jpg}{eps}{.jpg.bb}{`convert #1 eps:-}
   \DeclareGraphicsRule{.gif}{eps}{.gif.bb}{`convert #1 eps:-}
   \DeclareGraphicsRule{.tiff}{eps}{.tiff.bb}{`convert #1 eps:-}
   \DeclareGraphicsRule{.bmp}{eps}{.bmp.bb}{`convert #1 eps:-}
   \DeclareGraphicsRule{.png}{eps}{.png.bb}{`convert #1 eps:-}
\fi


\maketitle


\ifhtml
\chapter*{Front Matter\label{front}}
\fi


\begin{abstract}
\noindent
This paper describes two MicroCT datasets that Creatis has made publicly
available to the imaging community. These datasets were acquired using
Synchrotron Radiation, have resolutions of 1.4 microns, and size in the range of
six Gigabytes. The provided datasets were previously included in a study to develop
new 3D imaging methods to analyze physiological microcracks in
human trabecular bone at the micrometric scale.

This paper is a pure data contribution. The datasets described here have been
made available in the public MIDAS database. Here we describe the content of
the files to make easier for others to use this data as input for their own
research work.  This adheres to the fundamental principle that scientific
publications must facilitate \textbf{reproducibility} of the reported results.
\end{abstract}

\tableofcontents

\section{Introduction}
The concept of bone quality is increasingly considered to be an important factor
to explain bone fragility in addition to bone mass~\cite{Seeman2008}.
Bone quality is defined as a combination of factors that determine how well
the skeleton can resist fracture~\cite{Licata2009}. Among others, such
properties include bone microarchitecture, mineralization, bone turnover, the
quality of collagen, lacuno-canalicular system and microscopic damage.

Microscopic damage accumulates in bone tissue due to physiological loading and
mechanical stress occurring in daily life. Microdamage accumulation has been
shown to play a major role in the increased bone fragility associated with aging
and osteoporosis~\cite{Fazzalari2002}. It has been identified in the form of
microcracks, whose size, morphology and localization are strongly related to the
mechanical constraints applied to the bone~\cite{Fratzl2008}.

Although many works aiming to study the microdamaging process have been
conducted in animal models, recently data have been reported on human bone
samples~\cite{Fratzl2008}. Moreover, studies on human trabecular bone allow
analysis of microarchitecture simultaneously with bone-damaging processes,
since the localization, orientation and shape of trabeculae strongly affect the
characteristics and nature of this damage~\cite{Fazzalari2002}.

Ideally, bone microcracks should be evaluated in three dimensions with isotropic
and sufficiently high spatial resolution. While, contrast agents are being
developed for 3D observations with standard micro-CT devices~\cite{Leng2008},
the spatial resolution of these images and the capacity of contrast agents to
specifically bind to microcracks are still not sufficient to provide relevant
3D data on microcrack morphology. On the other hand, Synchrotron Radiation
Micro-Computerized Tomography (SR $\mu$-CT) possesses significant advantages
over standard micro-CT. A synchrotron source provides a high-flux, high-
intensity and monochromatic X-ray beam, allowing acquisition of quantitative
high-resolution 3D images with a high signal-to-noise ration~\cite{Salome1999}.
SR $\mu$-CT has already been used to study trabecular bone microarchitecture,
remodeling and local mineralization~\cite{Nuzzo2002}.

The datasets here introduced make part of a study to develop new 3D imaging
methods based on SR $\mu$-CT images to analyze physiological microcracks in
human trabecular bone at the micrometric scale. The precision of SR $\mu$-CT
in combination with image processing techniques allows to study the morphology
of microcracks. For further details on the developed methodology, we refer the
interested reader to the publication from Larrue~\textit{et al}~\cite{Larrue2011}.

\section{Acquisition Process}
\subsection{Sample Preparation}
Femoral head trabecular bone compartments were obtained from patients undergoing
total hip replacement for osteoarthritis on a study population consisting of
an equal number of male and female subjects between the ages of 77 and 80 years
old. Signed informed consent was obtained from all patients. 20 mm thick slices
of femoral heads were cut in the traverse plane by surgeons immediately after
surgery.

Sample preparation was performed following the guidelines from \cite{Davies2006}.
Cylinders, 10 mm diameter, were drilled with a diamond threphine from the original
slices and cut to a height of 5 mm using a Leica SP1600 Saw microtome. During the
cutting procedure, the bone was irrigated with sterile 0.9\% sodium chloride at
4\textcelsius\verb+ + %Any elegant solution to this matter?
to limit the-generated damage, remove bone chips, and prevent drying.
Next, bone cores were embedded in methylmethacrylate. Finally, pseudo-parallelepiped
samples (5x5x10mm) were prepared for SR $\mu$-CT imaging.

\subsection{Imaging}
Images were acquired on the SR $\mu$-CT setup developed on beamline ID19 at
the European Synchrotron Radiation Facility (ESRF), Grenoble, France~
\cite{Weitkamp2010}.
Samples were attached on stands adapted to the rotation stage and were placed as
close as possible to the camera to limit phase contrast. Crystal monochromators
were used to select a single X-ray energy, set to 24 keV. The X-ray beam
transmitted through the sample was acquired on a detector made up of a YaG
scintillator screen, an optical lens and a 2048 $\times$ 2048 CCD FReLoN camera.
Due to limited beam height at the selected energy, only a 2048 $\times$ 1440 ROI
was used on the detector. Since pixel size was set to 1.4 $\mu$m, the field of
view in the sample was of 2.8 $\times$ 2.8 $\times$ 2.0 mm$^3$. The sample was
positioned to allow imaging of its central core, thus excluding imaging of
microdamage in the sides of the specimen.

For each sample, 1999 radiographs were taken at different angles. A Filtered
Back Projection algorithm was used to obtain a reconstructed 3D volume. Due to the
large size of the reconstructed data, the resulting 3D volume was partitioned
into 6 different subvolumes. Table~\ref{table:summary} presents the main
parameter values used in the image acquisition protocol. Further details on the
complete protocol can be found in ~\cite{Larrue2011}.

\begin{table}[t]
\caption{Summary of dataset 1 acquisition parameters}
\begin{center}
 \begin{tabular}{l|l}
 \hline
 \multicolumn{1}{c}{\textbf{Parameter}} & \multicolumn{1}{c}{\textbf{Value}}\\
 \hline
 Monochromator & Multilayered \\
 Energy & 24 keV \\
 Pixel size & 1.4 $\mu$m \\
 Scan time & 26 minutes \\
 Count time & 0.5 s/projection \\
 Number of projections & 1999 over 360\textdegree \\
 Optic & magnification 20 \\
 Scintillator & YAG Ce 25 $\mu$m \\
 Detector & 2048 $\times$ 2048 CCD FReLoN HD2k \\
 Size of projections &  2048 $\times$ 1440 \\
 Number of reconstructed subvolumes & 6 \\
 Subvolume size & 2048 $\times$ 2048 $\times$ 256 (5 subvolumes)\\
                & 2048 $\times$ 2048 $\times$ 140 (1 subvolume)\\
 \hline
\end{tabular}
\end{center}
\label{table:summary}
\end{table}

\section{Dataset Description and Location}
We have contributed to the MIDAS Database with two datasets (denoted as Dataset 1
and 2) that have been used for bone microcrack analysis. In the following, we
give a short description of each dataset and we provide its location in the
MIDAS Database.

\subsection{Dataset 1}
\label{sec:data1}
Dataset 1 represents the original data obtained after tomographic reconstruction
using the acquisition parameters described in Table~\ref{table:summary}. Its size
is around 22 Gb. This dataset is named

\begin{itemize}
\item hunc34\_14\_a\_float.mhd
\item hunc34\_14\_a\_float.raw
\end{itemize}

Its identifier in MIDAS is

\url{http://hdl.handle.net/1926/1750}

Figure~\ref{fig:OriginalImage} presents an example of a slice extracted from
the original image.

\subsection{Dataset 2}
Due to the large size of the originally reconstructed volumes, datasets are
typically rescaled from floating point accuracy (32 bits per pixel) to an
unsigned char pixel type (8 bits per pixel). Such rescaling reduces the image
size to approximately 6 Gb. Dataset 2 is the rescaled version of the dataset
introduced in Section~\ref{sec:data1}
This dataset is named

\begin{itemize}
\item hunc34\_14\_a.mhd
\item hunc34\_14\_a.raw
\end{itemize}

Its identifier in MIDAS is

\url{http://hdl.handle.net/1926/1750}

\clearpage
\begin{figure}
\center
\begin{tabular}{c c}

\includegraphics[width=0.365\textwidth]{../../Testing/Temporary/ReadWriteTest_hunc34_14_a_Slice.png} &
\includegraphics[width=0.5\textwidth]{../../Testing/Temporary/Rendering3D_hunc34_14_a.png}\\
\end{tabular}

%\includegraphics[width=0.5\textwidth]{../../Testing/Temporary/Rendering3D_hunc34_14_a.png}
\itkcaption[Original Image]{Left. Middle slice from the original float image. Right. Volume rendering of the
dataset.}
\label{fig:OriginalImage}
\end{figure}

\subsection{Datasets Location}

The datasets are in the MIDAS Database:\\
\url{http://www.insight-journal.org/midas/community/view/33} \\
and have the unique identifier:
\url{http://hdl.handle.net/1926/1750}

\section{Licensing}

Along the spirit of contributing to the progress of the field, Creatis is
generously making these dataset publicly available under the terms of the Open
Data License.
\url{http://www.opendatacommons.org/licenses/by/}

This is a human-readable summary of the ODC-BY 1.0 license. Please see the disclaimer below.
You are free:\\

\begin{itemize}
\item To Share: To copy, distribute and use the database.
\item To Create: To produce works from the database.
\item To Adapt: To modify, transform and build upon the database.
\end{itemize}

As long as you:\\

\begin{itemize}
\item Attribute: You must attribute any public use of the database, or works
produced from the database, in the manner specified in the license. For any use
or redistribution of the database, or works produced from it, you must make
clear to others the license of the database and keep intact any notices on the
original database.
\end{itemize}


\subsection{Attribution}

In order to satisfy the Attribution requirement of the license, when using
these data, please acknowledge Creatis using the following text

\begin{verbatim}
This data was provided by Team 7 (responsible F. Peyrin) at Creatis laboratory.
The data was acquired with funds from the ESRF long term project entitled
'Synchrotron radiation micro-CT for the investigation of bone quality'.
\end{verbatim}

\section{Conclusions}
This paper has described two datasets that have been made publicly available
through the MIDAS Database. The first dataset  (~22 Gb volume) is directly
provided by the reconstruction program, while the second (~6 Gb volume) is the
obtained result after rescaling to 8 bits. These datasets were previously used in
the analysis of microcrack morphology through 3D image processing techniques. 
We refer the interested reader to the previous work~\cite{Larrue2011} on
microcrack detection for further details.

\section{Contact}

You can contact the authors at peyrin@esrf.fr.


%%%%%%%%%%%%%%%%%%%%%%%%%%%%%%%%%%%%%%%%%
%
%  Insert the bibliography using BibTeX
%
%%%%%%%%%%%%%%%%%%%%%%%%%%%%%%%%%%%%%%%%%

\bibliographystyle{plain}
\bibliography{InsightJournal}


\end{document}
