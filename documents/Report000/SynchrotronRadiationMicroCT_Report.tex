\documentclass{InsightArticle}

\usepackage[dvips]{graphicx}
\usepackage{color}
\usepackage{listings}
\usepackage{verbatim}
\usepackage{textcomp}

\definecolor{listcomment}{rgb}{0.0,0.5,0.0}
\definecolor{listkeyword}{rgb}{0.0,0.0,0.5}
\definecolor{listnumbers}{gray}{0.65}
\definecolor{listlightgray}{gray}{0.955}
\definecolor{listwhite}{gray}{1.0}


%%%%%%%%%%%%%%%%%%%%%%%%%%%%%%%%%%%%%%%%%%%%%%%%%%%%%%%%%%%%%%%%%%
%
%  hyperref should be the last package to be loaded.
%
%%%%%%%%%%%%%%%%%%%%%%%%%%%%%%%%%%%%%%%%%%%%%%%%%%%%%%%%%%%%%%%%%%
\usepackage[dvips,
bookmarks,
bookmarksopen,
backref,
colorlinks,linkcolor={blue},citecolor={blue},urlcolor={blue},
]{hyperref}


\title{Acquisition of Synchrotron Radiation micro-CT images for the investigation of bone
micro-cracks}


%
% NOTE: This is the last number of the "handle" URL that
% The Insight Journal assigns to your paper as part of the
% submission process. Please replace the number "1338" with
% the actual handle number that you get assigned.
%
\newcommand{\IJhandlerIDnumber}{3261}

\lstset{frame = tb,
       framerule = 0.25pt,
       float,
       fontadjust,
       backgroundcolor={\color{listlightgray}},
       basicstyle = {\ttfamily\footnotesize},
       keywordstyle = {\ttfamily\color{listkeyword}\textbf},
       identifierstyle = {\ttfamily},
       commentstyle = {\ttfamily\color{listcomment}\textit},
       stringstyle = {\ttfamily},
       showstringspaces = false,
       showtabs = false,
       numbers = left,
       numbersep = 6pt,
       numberstyle={\ttfamily\color{listnumbers}},
       tabsize = 2,
       language=[ANSI]C++,
       floatplacement=!h
       }

\release{1.10}

\author{Maria A. Zuluaga$^{1,2}$, Aymeric Larrue$^{3}$, Aline Rattner$^{4}$, \\Laurence Vico$^{4}$, Fran\c{c}oise Peyrin$^{1,2}$}
\authoraddress{$^{1}$CREATIS; Universit\'{e} de Lyon; Universit\'{e} Lyon 1; INSA-Lyon; CNRS UMR5220; INSERM U630; F-69621 Villeurbanne, France\\
$^{2}$ European Synchrotron Radiation Facility, Grenoble, France\\
$^{3}$Institute of Biomedical Engineering, Department of Engineering Science, University of Oxford, \\Oxford, UK\\
$^{4}$LBTO, Université de Lyon, St-Etienne, F-42023 France ; Inserm, U890, \\Saint-Etienne F-42023
St. Etienne, France}

\begin{document}


%
% Add hyperlink to the web location and license of the paper.
% The argument of this command is the handler identifier given
% by the Insight Journal to this paper.
%
\IJhandlefooter{\IJhandlerIDnumber}


\ifpdf
\else
   %
   % Commands for including Graphics when using latex
   %
   \DeclareGraphicsExtensions{.eps,.jpg,.gif,.tiff,.bmp,.png}
   \DeclareGraphicsRule{.jpg}{eps}{.jpg.bb}{`convert #1 eps:-}
   \DeclareGraphicsRule{.gif}{eps}{.gif.bb}{`convert #1 eps:-}
   \DeclareGraphicsRule{.tiff}{eps}{.tiff.bb}{`convert #1 eps:-}
   \DeclareGraphicsRule{.bmp}{eps}{.bmp.bb}{`convert #1 eps:-}
   \DeclareGraphicsRule{.png}{eps}{.png.bb}{`convert #1 eps:-}
\fi


\maketitle


\ifhtml
\chapter*{Front Matter\label{front}}
\fi


\begin{abstract}
\noindent
This paper describes two MicroCT datasets that Creatis has made publicly
available to the imaging community. These datasets were acquired using
Synchrotron Radiation, have resolutions of 1.4 microns, and size in the range of
six GigaBytes.

This paper is a pure data contribution. The datasets described here have been
made available in the public MIDAS database. Here we describe the content of
the files to make easier for others to use this data as input for their own
research work.  This adheres to the fundamental principle that scientific
publications must facilitate \textbf{reproducibility} of the reported results.
\end{abstract}

\tableofcontents

\section{Introduction}

The concept of bone quality is increasingly considered to be an important factor
to explain bone fragility in addition to bone mass. Studies on trabecular 
bone allow analysis of microarchitecture simultaneously with bone-damaging
processes, since the localization, orientation and shape of trabeculae strongly
affect the characteristics and nature of this damage.

While, contrast agents are being developed for 3D observations with standard
micro-CT device, the spatial resolution of these images and the capacity of
contrast agents to specifically bind to microcracks are still not sufficient
to provide relevant 3D data on microcrack morphology. On the other hand,
Synchrotron Radiation Micro-Computerized Tomography (SR $\mu$-CT) possesses
significant advantages over standard micro-CT.

The datasets we are here introducing made part of a study in order to develop 
 new 3D imaging methods based on SR $\mu$-CT images to analyze physiological
microcracks in human trabecular bone at the micrometric scale. The precision
of SR $\mu$-CT and combined with image processing techniques allow to study
 the actual morphology of microcracks. For further details on the developed
methodology, we refer the reader to~\cite{Larrue2011}. 

\section{Acquisition Process}
\subsection{Sample Preparation}
Femoral head trabecular bone compartments were obtained from patients undergoing  
total hip replacement for osteoarthritis. 

Sample preparation was performed following the guidelines from \cite{Davies2006}. 
Cylinders, 10 mm diameter, were drilled with a diamond threphine from the original 
slices and cut to a height of 5 mm using a Leica SP1600 Saw microtome. During the 
cutting procedure, the bone was irrigated with sterile 0.9\% sodium chloride at 
4\textcelsius  %Do you know how to fix this?
 to limit the-gnerated damage, remove bone chips, and prevent drying.
Next, bone cores were embedded in methylmethacrylate. Finally, pseudo-parallelepiped 
samples (5x5x10mm) were prepared for SR $\mu$-CT imaging.  

\subsection{Imaging}
Images were acquired on the SR $\mu$-CT setup developed on beamline ID19 at 
the European Synchrotron Radiation Facility (ESRF), Grenoble, France~\cite{Weitkamp2010}. 
Samples were attached on stands adapted to the rotation stage and were placed as 
close as possible to the camera to limit phase contrast.  

Further details on the whole acquisition protocol can be found in \cite{Larrue2011}. 

\section{Datasets Location}

The datasets are in the MIDAS Database:
\url{http://www.insight-journal.org/midas/community/view/33}

and have the unique identifier:
\url{http://hdl.handle.net/1926/1750}


\section{Dataset 1}


This dataset is named

\begin{itemize}
\item hunc34\_14\_a\_float.mhd
\item hunc34\_14\_a\_float.raw
\end{itemize}

Its identifier in MIDAS is

% FIXME
\url{http://hdl.handle.net/1926/1750}

This dataset represents the original data that is obtained after tomographic
reconstruction. Its size around 22 Gb. Table~\ref{table:summary} summarizes
the principal acquisition parameters of this particular dataset.

\begin{table}[t]
\caption{Summary of dataset 1 acquisition parameters}
\begin{center}
 \begin{tabular}{l|l}
 \hline
 \multicolumn{1}{c}{\textbf{Parameter}} & \multicolumn{1}{c}{\textbf{Value}}\\
 \hline
 Monochromator & Multilayerer \\
 Energy & 24 keV \\
 Pixel size & 1.4 $\mu$m \\
 Scan time & 26 minutes \\
 Count time & 0.5 s/projection \\
 Number of projections & 1999 over 360\textdegree \\
 Optic & magnification 20 \\
 Scintillator & YAG Ce 25 $\mu$m \\
 Detector & 2048 $\times$ 2048 CCD FReLoN HD2k \\
 Size of projections &  2048 $\times$ 1440 \\
 Number of reconstructed subvolumes & 6 \\
 Subvolume size & 2048 $\times$ 2048 $\times$ 256 (5 subvolumes)\\
                & 2048 $\times$ 2048 $\times$ 140 (1 subvolume)\\
 \hline
\end{tabular}
\end{center}
\label{table:summary}
\end{table} 

\section{Dataset 2}

This dataset is named

\begin{itemize}
\item hunc34\_14\_a.mhd
\item hunc34\_14\_a.raw
\end{itemize}

Its identifier in MIDAS is

% FIXME
\url{http://hdl.handle.net/1926/1750}

The dataset is the result of scaling the original float image (32 bits per pixel)
to an image with an unsigned char pixel type (8 bits per pixel). The latter
reduces the image size to approximately 6 Gb.

\section{Licensing}

Along the spirit of contributing to the progress of the field, Creatis is
generously making these dataset publicly available under the terms of the Open
Data License.
\url{http://www.opendatacommons.org/licenses/by/}

This is a human-readable summary of the ODC-BY 1.0 license. Please see the disclaimer below.
You are free:\\

\begin{itemize}
\item To Share: To copy, distribute and use the database.
\item To Create: To produce works from the database.
\item To Adapt: To modify, transform and build upon the database.
\end{itemize}

As long as you:\\

\begin{itemize}
\item Attribute: You must attribute any public use of the database, or works
produced from the database, in the manner specified in the license. For any use
or redistribution of the database, or works produced from it, you must make
clear to others the license of the database and keep intact any notices on the
original database.
\end{itemize}


\subsection{Attribution}

In order to satisfy the Attribution requirement of the license, when using
these data, please acknowledge Creatis using the following text

\begin{verbatim}
This data was provided by Team 7 under the direction of Fran\c{c}oise Peyrin 
at Creatis laboratory .The data was acquired with funds from the ESRF long
term project entitled 'Synchrotron radiation micro-CT for the investigation
of bone quality'.
\end{verbatim}

\section{Contact}

You can contact the authors at peyrin@esrf.fr


%%%%%%%%%%%%%%%%%%%%%%%%%%%%%%%%%%%%%%%%%
%
%  Insert the bibliography using BibTeX
%
%%%%%%%%%%%%%%%%%%%%%%%%%%%%%%%%%%%%%%%%%

\bibliographystyle{plain}
\bibliography{InsightJournal}


\end{document}
