\documentclass{InsightArticle}

\usepackage[dvips]{graphicx}
\usepackage{color}
\usepackage{listings}

\definecolor{listcomment}{rgb}{0.0,0.5,0.0}
\definecolor{listkeyword}{rgb}{0.0,0.0,0.5}
\definecolor{listnumbers}{gray}{0.65}
\definecolor{listlightgray}{gray}{0.955}
\definecolor{listwhite}{gray}{1.0}


%%%%%%%%%%%%%%%%%%%%%%%%%%%%%%%%%%%%%%%%%%%%%%%%%%%%%%%%%%%%%%%%%%
%
%  hyperref should be the last package to be loaded.
%
%%%%%%%%%%%%%%%%%%%%%%%%%%%%%%%%%%%%%%%%%%%%%%%%%%%%%%%%%%%%%%%%%%
\usepackage[dvips,
bookmarks,
bookmarksopen,
backref,
colorlinks,linkcolor={blue},citecolor={blue},urlcolor={blue},
]{hyperref}


\title{Large Image Streaming using ITKv4}


%
% NOTE: This is the last number of the "handle" URL that
% The Insight Journal assigns to your paper as part of the
% submission process. Please replace the number "1338" with
% the actual handle number that you get assigned.
%
\newcommand{\IJhandlerIDnumber}{3263}

\lstset{frame = tb,
       framerule = 0.25pt,
       float,
       fontadjust,
       backgroundcolor={\color{listlightgray}},
       basicstyle = {\ttfamily\footnotesize},
       keywordstyle = {\ttfamily\color{listkeyword}\textbf},
       identifierstyle = {\ttfamily},
       commentstyle = {\ttfamily\color{listcomment}\textit},
       stringstyle = {\ttfamily},
       showstringspaces = false,
       showtabs = false,
       numbers = left,
       numbersep = 6pt,
       numberstyle={\ttfamily\color{listnumbers}},
       tabsize = 2,
       language=[ANSI]C++,
       floatplacement=!h
       }

\release{1.10}

\author{Maria A. Zuluaga$^{1,2}$, Luis Ib\'{a}\~{n}ez$^{3}$, Fran\c{c}oise Peyrin$^{1,2}$}
\authoraddress{$^{1}$CREATIS; Universit\'{e} de Lyon; Universit\'{e} Lyon 1; INSA-Lyon; CNRS UMR5220; INSERM U630; F-69621 Villeurbanne, France\\
	           $^{2}$ European Synchrotron Radiation Facility, Grenoble, France\\
               $^{3}$Kitware Inc., Clifton Park, NY}

\begin{document}


%
% Add hyperlink to the web location and license of the paper.
% The argument of this command is the handler identifier given
% by the Insight Journal to this paper.
%
\IJhandlefooter{\IJhandlerIDnumber}


\ifpdf
\else
   %
   % Commands for including Graphics when using latex
   %
   \DeclareGraphicsExtensions{.eps,.jpg,.gif,.tiff,.bmp,.png}
   \DeclareGraphicsRule{.jpg}{eps}{.jpg.bb}{`convert #1 eps:-}
   \DeclareGraphicsRule{.gif}{eps}{.gif.bb}{`convert #1 eps:-}
   \DeclareGraphicsRule{.tiff}{eps}{.tiff.bb}{`convert #1 eps:-}
   \DeclareGraphicsRule{.bmp}{eps}{.bmp.bb}{`convert #1 eps:-}
   \DeclareGraphicsRule{.png}{eps}{.png.bb}{`convert #1 eps:-}
\fi


\maketitle


\ifhtml
\chapter*{Front Matter\label{front}}
\fi


\begin{abstract}
\noindent
This document illustrates how to process large images (6 and 23 Gigabytes in
size) by taking advantage of the streaming capabilities of the Insight Toolkit
ITK.  Here we illustrate two scenarios: (a) the case when the image itself is
larger than the computer's RAM, (b) the case when the image is large but still
can fit in the computer's RAM. This report is intended to serve as a tutorial
on how to take advantage of this memory management capabilities of ITK.

This paper is accompanied with the source code, input data, parameters and
output data that we used for validating the algorithm described in this paper.
This adheres to the fundamental principle that scientific publications must
facilitate \textbf{reproducibility}~\cite{Stodden2009} of the reported results.
\end{abstract}

\tableofcontents

\section{Introduction}

Large images are becoming ubiquitous in many research fields. They are
particularly common in microscopy, remote sensing and computer vision.  Image
sizes are growing at a higher rate than the size of computer memory and
therefore there is a great interest in software methods that allow to process
images by partitioning in pieces that can temporarily be fit into memory. Such a 
process is commonly denoted as streaming.

In this technical report we illustrate how this can be done with the current
infrastructure of the Insight Toolkit (ITK). For this matter,
we make use of two large images (6 and 23 Gigabytes in size) from the
Creatis collection of the MIDAS database~\cite{Zuluaga2011a}.

\section{Implementation}
The example we develop illustrates the pipeline executed in order to segment
osteocytes from trabecular bone images. Osteocytes appear as small dark regions
inside white larger structures (Figure XXX). A simple pipeline to extract these
dark regions consists of: 1) Image binarization, 2) Hole-filling on the binarized
image and 3) Subtraction of the hole-filled and the binarized images.
In the following, we present the different stages of the pipeline and how the whole
process can be streamed. 

\subsection{Reading and Writing}
A key requirement in order to perform such task through the partitioning of
images is that the \verb+ImageIO+ classes involved in the pipeline support
streamed reading and writing. Otherwise, the entire image will be buffered
in memory~\cite{Lowekamp2010}. 

The following code illustrates the reading of a large image that is then written
into another file. While this is the simplest case that can be illustrated, it
contains the core elements that are required to perform streaming.

First we include the headers:

\begin{center}
\lstinputlisting[linerange={21-23}]{../../src/ImageReadStreamWrite.cxx}
\end{center}

Then, we instantiate the reader and writer.

\begin{center}
\lstinputlisting[linerange={46-47}]{../../src/ImageReadStreamWrite.cxx}
\end{center}

In order to trigger the use of streaming, it is necessary to specify to the
writer into how many blocks the image should be partitioned. In that sense, the
most important line in the streaming process is:

\begin{center}
\lstinputlisting[linerange={57-57}]{../../src/ImageReadStreamWrite.cxx}
\end{center}

Finally, we use the standard try / catch block that calls the Update method and
triggers the whole process. 

\begin{center}
\lstinputlisting[linerange={68-76}]{../../src/ImageReadStreamWrite.cxx}
\end{center}

\subsection{Binary Thresholding}
The first filter used in the segmentation pipeline is a binary thresholding that
allows to differentiate the osteocytes from the rest of the image. In this example,
we connected a binary thresholding filter between the reader and the writer. 

\begin{center}
\lstinputlisting[linerange={48-59}]{../../src/BinaryThresholdImageFilter.cxx}
\end{center}

As can be seen from the following code, there is no significant difference
between the previous example and this one in order to activate the streaming.
The streaming process is still driven by the writer: 

\begin{center}
\lstinputlisting[linerange={79-79}]{../../src/BinaryThresholdImageFilter.cxx}
\end{center}

\subsection{Noise Elimination and Filling-Up Holes}
It is important to differentiate the streaming concept from multithreading. While
streaming refers to the process of sequentially processing sub-regions part of
the largest possible region) of an image through the pipeline~\cite{Lowekamp2010},
multithreading refers to the possibility of making use of the multiple cores of
a computer to do the processing.

In ITK, not all filters supporting streaming support the use of multi-threading.
Such is the case of ITK's morphological filters, which are commonly used for hole
filling. Therefore, we have selected to use \verb+itkVotingBinaryHoleFillingImageFilter+,
for the hole filling task, instead of the morphological filters, since the former
supports both streaming and multithreading.

The key elements required to make us of  \verb+itkVotingBinaryHoleFillingImageFilter+
are shown in the following:

\begin{center}
\lstinputlisting[linerange={44-65}]{../../src/VotingBinaryHoleFillingImageFilter.cxx}
\end{center}

Depending on the foreground and background values that are provided to the filter,
it can perform hole filling or noise removal from the background (see Figure XXX).
We have used it here for both purposes.

We recall again in the fact that there is only a single code line that
need to be included to activate the streaming: 

\begin{center}
\lstinputlisting[linerange={74-74}]{../../src/VotingBinaryHoleFillingImageFilter.cxx}
\end{center}

On the other hand, nothing needs to be added in order to activate the multi-threading.
If the filter allows the use of multiple cores, it will make use of them.

\subsection{Subtraction}
The final stage of the pipeline consists in the subtraction of  a noise-free
version of the binarized image from the hole-filled image. The implementation of
this stage is straightforward and it has no additional difficulty in what respects
to the streaming. For the sake of completeness, we limit ourselves to just mention
this stage but, we do not include code example in the report.

\section{Results}
We have evaluated the pipeline using two different images from the Creatis collection
in the MIDAS database\cite{Zuluaga2011a} with respective sizes of 6 and 23 Gigabytes.
Figures~\ref{fig:ImageOne} and \ref{fig:ImageTwo} show the results obtained at different
stages of the pipeline. 

\begin{figure}
\center
\begin{tabular}{c c c}
\includegraphics[width=0.3\textwidth]{../../Testing/Temporary/ReadWriteTest_hunc34_14_a_Slice.png} &
\includegraphics[width=0.3\textwidth]{../../Testing/Temporary/BinaryThresholdTest_hunc34_14_a_Slice.png} &
\includegraphics[width=0.3\textwidth]{../../Testing/Temporary/VotingHoleFillingTest_01_hunc34_14_a_Slice.png} \\
\includegraphics[width=0.3\textwidth]{../../Testing/Temporary/VotingHoleFillingTest_02_hunc34_14_a_Slice.png} &
\includegraphics[width=0.3\textwidth]{../../Testing/Temporary/VotingHoleFillingTest_04_hunc34_14_a_Slice.png} &
\includegraphics[width=0.3\textwidth]{../../Testing/Temporary/SubtractImageTest_hunc34_14_a_Slice.png} \\
\end{tabular}

\itkcaption[Pipeline execution on image 1]{First row. Left. Original image. Center.
Binarized image. Right. Image after noise removal. Second row. Left. Hole-filling after one
iteration. Center. Hole-filling after three iterations. Right. Subtracted image.}
\label{fig:ImageOne}
\end{figure}

\begin{figure}
\center
\begin{tabular}{c c c}
\includegraphics[width=0.3\textwidth]{../../Testing/Temporary/ReadWriteTest_hunc34_14_a_float_Slice.png} &
\includegraphics[width=0.3\textwidth]{../../Testing/Temporary/BinaryThresholdTest_hunc34_14_a_float_Slice.png} &
\includegraphics[width=0.3\textwidth]{../../Testing/Temporary/VotingHoleFillingTest_01_hunc34_14_a_float_Slice.png} \\
\includegraphics[width=0.3\textwidth]{../../Testing/Temporary/VotingHoleFillingTest_02_hunc34_14_a_float_Slice.png} &
\includegraphics[width=0.3\textwidth]{../../Testing/Temporary/VotingHoleFillingTest_04_hunc34_14_a_float_Slice.png} &
\includegraphics[width=0.3\textwidth]{../../Testing/Temporary/SubtractImageTest_hunc34_14_a_float_Slice.png} \\
\end{tabular}
\itkcaption[Pipeline execution on image 1]{First row. Left. Original image. Center.
Binarized image. Right. Image after noise removal. Second row. Left. Hole-filling after one
iteration. Center. Hole-filling after three iterations. Right. Subtracted image.}
\label{fig:ImageTwo}
\end{figure}

We should mention that, as can be seen from figures~\ref{fig:ImageOne} and
\ref{fig:ImageTwo}, the hole-filling process requires the iterative execution of
the voting filter. While ITK provides a \verb+VotingBinaryIterativeHoleFillingImageFilter+
that iterates until all the image holes are filled, we select to use the simpler
\verb+VotingBinaryHoleFillingImageFilter+ since it permits the use of streaming
and multithreading.

The whole pipeline was tested on an Dual Core Intel Centrino with 4Gb in RAM. For the image of 6GB (image 1),
the data was partitioned into 6 different blocks. For the 23 Gb image (image 2), the data was partitioned into
10 different blocks. Given $f$ the number of filters involved
in a processing pipeline and $s$ the size in (bytes) of a slice, a simple rule that can be
used to estimate the desired number of partitions $p$,  should be

\begin{equation}
  \frac{f*s}{p} < RAM
\end{equation}

Afterwards, a trade-off between the speed of the process and the number of partitions
should be evaluated. As the number of partitions increases, the pipeline execution will
take longer, unless a sufficient number of CPU cores is available. As an example,
Table~\ref{table:times} summarizes the compuational times of each stage of the pipeline
with the two evaluated images.

\begin{table}
\caption{Computational times for each of the pipeline stages.}
\begin{center}
 \begin{tabular}{l|ll}
 \hline
 & \multicolumn{1}{c}{\textbf{Image 1}} & \multicolumn{1}{c}{\textbf{Image 2}}\\
 \hline
Read and Write & 93.26 & 141.137\\
Binarization & 129.79 & 181.06 \\
Noise removal & 782.92 & 1075.76 \\
Hole-filling (iteration 1) & 1147.51 & 2399.96 \\
Hole-filling (iteration 1) & 1665.01 & 2436.57\\
Hole-filling (iteration 1) & 1656.96 & 2414.11\\
Subtraction & 97.8127 & 223.576\\
 \hline
\end{tabular}
\end{center}
\label{table:times}
\end{table}

\section{Conclusions}
In this paper we have illustrated, through a simple example, how can the streaming
and multithreading capabilities of the Insight Toolkit can be exploited so that
large images can be processed.  

\bibliographystyle{plain}
\bibliography{InsightJournal}


\end{document}
